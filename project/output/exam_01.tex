
\documentclass[12pt]{ctexart}
\usepackage[a4paper,margin=2cm]{geometry}
\usepackage[answers=false]{exam-zh}
\usepackage{enumitem}
\usepackage{xeCJK}
\setCJKmainfont{Noto Serif CJK SC}

% Toggle answers with \printanswers
% The answers version will add \printanswers after \begin{document}

\begin{document}
\vspace*{-2em}
\begin{center}
  {\LARGE 法律法规综合测试卷}\\[4pt]
  考试时间:120分钟 \quad 满分:150分
\end{center}
\vspace{0.5em}

\section*{一、判断题}
\begin{questions}
\question {}\ifprintanswers\par\textbf{答案:}对\fi
\question {}\ifprintanswers\par\textbf{答案:}对\fi
\end{questions}
\section*{二、单选题}
\begin{questions}
\question 依据第1条,下列哪一项是正确的?
不得侵犯他人合法权益。
\begin{enumerate}[label=\Alph*.]
\item 关于“义务”的表述不符合该法条
\item 关于“应当”的表述符合该法条
\item 关于“不得”的表述不符合该法条
\item 关于“责任”的表述不符合该法条
\end{enumerate}
\ifprintanswers\par\textbf{答案:}B\fi
\question 依据第1条,下列哪一项是正确的?
公民依法享有权利并履行义务。
\begin{enumerate}[label=\Alph*.]
\item 关于“不得”的表述不符合该法条
\item 关于“应当”的表述不符合该法条
\item 关于“责任”的表述不符合该法条
\item 关于“义务”的表述符合该法条
\end{enumerate}
\ifprintanswers\par\textbf{答案:}D\fi
\end{questions}
\section*{三、多选题}
\begin{questions}
\question 依据第1条,下列哪些项是正确的?
公民依法享有权利并履行义务。
\begin{enumerate}[label=\Alph*.]
\item 关于“可以”的表述不符合该法条
\item 与“义务”相关的规定符合该法条
\item 与“不得”相关的规定符合该法条
\item 关于“不得”的表述不符合该法条
\item 与“责任”相关的规定符合该法条
\end{enumerate}
\ifprintanswers\par\textbf{答案:}B,C,E\fi
\question 依据第2条,下列哪些项是正确的?
应当公开、公正、公平。
\begin{enumerate}[label=\Alph*.]
\item 关于“不得”的表述不符合该法条
\item 关于“可以”的表述不符合该法条
\item 与“可以”相关的规定符合该法条
\item 关于“责任”的表述不符合该法条
\item 与“应当”相关的规定符合该法条
\end{enumerate}
\ifprintanswers\par\textbf{答案:}C,E\fi
\end{questions}
\section*{四、填空题}
\begin{questions}
\question 行政机关可以依照法定权限和程序实施行政管理。\textbackslash{}rule\{2cm\}\{0.4pt\}公开、公正、公平。
\ifprintanswers\par\textbf{答案:}应当\fi
\question 违反规定的,\textbackslash{}rule\{2cm\}\{0.4pt\}承担相应责任。
\ifprintanswers\par\textbf{答案:}应当\fi
\end{questions}
\section*{五、简答题}
\begin{questions}
\question 简述第1条的主要内容或立法目的。
\ifprintanswers\par\textbf{要点:}公民依法享有权利并履行义务。不得侵犯他人合法权益。违反规定的,应当承担相应责任。\fi
\question 简述第2条的主要内容或立法目的。
\ifprintanswers\par\textbf{要点:}行政机关可以依照法定权限和程序实施行政管理。应当公开、公正、公平。\fi
\end{questions}

\end{document}

